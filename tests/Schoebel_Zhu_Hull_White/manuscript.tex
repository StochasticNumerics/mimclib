\documentclass[11pt]{amsart}
\usepackage{geometry}                % See geometry.pdf to learn the layout options. There are lots.
\geometry{a4paper}                   % ... or a4paper or a5paper or ... 
%\geometry{landscape}                % Activate for for rotated page geometry
%\usepackage[parfill]{parskip}    % Activate to begin paragraphs with an empty line rather than an indent
\usepackage{graphicx}
\usepackage{amssymb, amsmath, amsthm}
\usepackage{natbib}
\usepackage{epstopdf}
\usepackage{enumerate}
\usepackage{textcomp}
\usepackage{mathrsfs }
\usepackage[section]{algorithm}
\usepackage{algorithmic}
\usepackage{tikz}
\usepackage{pgfplots}
\usepgfplotslibrary{patchplots}
\usepackage{subfigure}
\usepackage{hyperref}
\usepackage{comment}
\usepackage{multirow}

%line suggested by my compiler (Fabian)
\pgfplotsset{compat=1.9}

% put figures at the end of the document, while I am writing is very annoying to have the figures in the middle of the text! 
% \usepackage[nomarkers,figuresonly]{endfloat}


\DeclareGraphicsRule{.tif}{png}{.png}{`convert #1 `dirname #1`/`basename #1 .tif`.png}

% \theoremstyle{theorem}
\newtheorem{theorem}{Theorem}[section]
\newtheorem{lemma}[theorem]{Lemma}
\newtheorem{proposition}[theorem]{Proposition}
\newtheorem{corollary}[theorem]{Corollary}
\newtheorem{point}[theorem]{}
\newtheorem{assumption}[theorem]{Assumption}


\theoremstyle{definition}
\newtheorem{definition}[theorem]{Definition}
\newtheorem{example}[theorem]{Example}
\newtheorem{notation}[theorem]{Notation}
\newtheorem{remark}[theorem]{Remark}
\newtheorem{condition}{Condition}

\newcommand{\F}{\operatorname{\mathcal{F}}}
\newcommand{\iF}{\operatorname{\mathcal{F}^{-1}}}

\newcommand{\Loperator}{\mathcal{L}}
%\newcommand{\R}{\mathbf{R}}
\newcommand{\Z}{\mathbf{Z}}

\newcommand{\dd}{\, \textrm{d}}
\newcommand{\e}{\textrm{exp}}
\newcommand{\fFrequency}{ f_{\omega_0}}
\newcommand{\fQuadrature}{ f_{\Delta}}

\newcommand{\I}{\mathbf{I}}

\newcommand{\ErrorFrequency}{\mathcal{E_F}}
\newcommand{\ErrorQuadrature}{\mathcal{E_Q}}
\newcommand{\Error}{\mathcal{E}}
\newcommand{\EF}{\ErrorFrequency}
\newcommand{\EQ}{\ErrorQuadrature}
\newcommand{\E}{\Error}
\newcommand{\est}{\overline {\mathcal{E}}}
\newcommand{\estFrequency}{\est_F}
\newcommand{\estQuadrature}{\est_Q}


\newcommand{\lx}{\mathcal{L}^{X}}
\newcommand{\ls}{\mathcal{L}^{S}}
\newcommand{\R}{\mathbb{R}}
\newcommand{\C}{\mathbb{C}}
\newcommand{\prob}[2]{\pi_{{#1,#2}}}
\newcommand{\lam}[1]{\lambda_#1}
\newcommand{\trials}[2]{\theta_{#1,#2}}
\newcommand{\damp}{\alpha}

\newcommand{\cgmyY}{Y}
\newcommand{\cgmyM}{\eta_+}
\newcommand{\cgmyG}{\eta_-}
\newcommand{\cgmyC}{K}

\newcommand{\cgmyexp}{\cgmyY}

\newcommand{\fa}{f_{\damp}}
\newcommand{\fadisc}[1]{f_{\damp, #1}}
\newcommand{\fdisc}[1]{f_{#1}}
\newcommand{\ga}{g_{\damp}}
\newcommand{\ha}{h_{\damp}}
\newcommand{\ft}[1]{\hat{#1}}
\newcommand{\kft}[1]{\tilde{#1}}
\newcommand{\ift}[1]{\iF\left[#1\right]}
\newcommand{\Lnorm}{\textrm{L}}
\newcommand{\real}[1]{\mathrm{Re}\left[#1\right]}
\newcommand{\im}[1]{\mathrm{Im}\left[#1\right]}
\newcommand{\step}{\Delta\omega}
\newcommand{\strip}[1]{A_{#1}}
\newcommand{\pureJumpProcessConstant}{C(\nu)}
\newcommand{\ffreq}[1]{\nu_{#1}}


\newcommand{\charExp}[1]{\Psi \parent{{#1}}}
\newcommand{\charFun}[2]{\varphi_{#1} \parent{{#2}}}

\newcommand{\cutoff}{\omega_{\mathrm{\max}}}
\newcommand{\strike}{k}

\newcommand{\pprob}[1]{}

\newcommand{\fnka}{f_{\damp}^{N,\cutoff}}
\newcommand{\ftnka}{\tilde f_{\damp}^{N,\cutoff}}

\newcommand{\pnka}{\Pi_{\damp}^{N,\cutoff}}

\newcommand{\fft}[2]{\mathrm{FFT}_{{#1}} \parent{{#2} }}
\newcommand{\tol}{\epsilon_{\mathrm T}}

\newcommand{\cutoffFrequency}{\omega_{\textrm{max}}}

\newcommand{\uremark}[2]{\underbrace{{#1}}_{{#2}}}

% Juho macros
\newcommand{\partder}[2]{\frac{\partial #1}{\partial #2}}
\newcommand{\der}[1]{\partial_{ #1}}
\newcommand{\ket}[1]{\left | {#1} \right \rangle }
\newcommand{\refstate}[0]{\textbf{0}}
\newcommand{\vacket}[0]{\ket{\refstate}}
\newcommand{\vacbra}[0]{\bra{\refstate}}
\newcommand{\bra}[1]{\left \langle {#1} \right | }
\newcommand{\ketbra}[1]{\ket{#1} \bra{#1}}
\newcommand{\innerprod}[2]{\left \langle {{#1} | {#2}} \right \rangle}
\newcommand{\uket}[0]{\ket{\uparrow}}
\newcommand{\dket}[0]{\ket{\downarrow}}
\newcommand{\ubra}[0]{\bra{\uparrow}}
\newcommand{\dbra}[0]{\bra{\downarrow}}
\newcommand{\hadj}[1]{{#1}^{\dagger}}
\newcommand{\conj}[1]{{#1}^{\ast}}
\newcommand{\cosine}[1]{\mathrm{cos}\left ( {#1}\right )}
\newcommand{\sine}[1]{\mathrm{sin}\left ( {#1}\right )}
\newcommand{\cosinep}[2]{\mathrm{cos}^{#2}\left ( {#1}\right )}
\newcommand{\sinep}[2]{\mathrm{sin}^{#2}\left ( {#1}\right )}
\newcommand{\expf}[1]{\mathrm{exp}\left ( {#1}\right )}
\newcommand{\expfs}[1]{e^{#1}}
\newcommand{\determinant}[1]{\mathrm{det}\left ( {#1}\right )}
\newcommand{\trace}[1]{\mathrm{Tr}\left ( {#1}\right )}
\newcommand{\parent}[1]{\left( {#1} \right)}
\newcommand{\aver}[1]{ \left\langle  {#1}  \right\rangle }
\newcommand{\absval}[1]{\left| {#1} \right|}
\newcommand{\sset}[1]{\left\lbrace {#1} \right\rbrace }
\newcommand{\iden}[1]{\mathbf{1}_{#1}}
\newcommand{\ssum}[2]{\displaystyle\sum\limits_{#1}^{#2}}
\newcommand{\pprod}[2]{\displaystyle\prod\limits_{#1}^{#2}}
\newcommand{\commut}[2]{\left[ {#1} , {#2} \right]}
\newcommand{\acommut}[2]{\left\lbrace  {#1} , {#2} \right\rbrace }
\newcommand{\spann}[1]{\mathrm{span} \parent{{#1}}}
\newcommand{\ttrace}[2]{\mathrm{Tr}_{#1} \parent{#2}}
\newcommand{\logt}[1]{\mathrm{log_2} \parent{{#1}}}
\newcommand{\hilbert}[1]{\mathcal{H}_{{#1}}}
\newcommand{\genus}{g}
\newcommand{\gsfun}[0]{\eta_{GS}}
\newcommand{\vvec}[1]{\textbf{{#1}}}
\newcommand{\kk}[0]{\vvec{k}}
\newcommand{\tsection}[1]{\newpage \section{{#1}}}
\newcommand{\grad}[0]{\nabla}
\newcommand{\divv}[0]{\nabla \dot}
\newcommand{\curl}[0]{\nabla \times}
\newcommand{\vvar}[1]{\mathrm{var} \parent{{#1}}}
\newcommand{\bigo}[1]{\mathcal O \parent{{#1}}}
\newcommand{\smallo}[1]{ o \parent{{#1}}}
\def\d{\textrm{d}}

\newcommand{\cost}[1]{C_{\mathrm{#1}}}

\newcommand{\qt}[0]{\tilde q}
%\newcommand{\e}[1]{\times 10^{#1}}
\newcommand{\supremum}[1]{\mathop{\mathrm{sup}}_{{#1}}}
\newcommand{\nnorm}[2]{\absval{\absval{{#1}}}_{#2}}
\newcommand{\problem}[1]{\newpage \section*{{#1}}}
\newcommand{\hittime}[0]{\sigma_{\bar{\mathcal E}}}
\newcommand{\probb}[1]{\mathrm{P} \parent{{#1}}}
\newcommand{\expp}[1]{\mathrm{E} \parent{{#1}}}

\newcommand{\kaustaddress}[0]{King Abdullah University of Science and Technology, Thuwal, Kingdom of Saudi Arabia}
\newcommand{\indicatorfun}[3]{\mathbf{1}_{[{#1},{#2}]} \parent{{#3}}}
\newcommand{\indicator}{\mathbf{1}}
\newcommand{\binaryg}[1]{g_{\mathrm{bin}} \parent{#1}}
\newcommand{\binaryf}[1]{f_{\mathrm{bin}} \parent{#1}}
\newcommand{\fbinaryf}[1]{\hat{f}_{\mathrm{bin}} \parent{#1}}

\newcommand{\callg}[1]{g_{\mathrm{call}} \parent{#1}}
\newcommand{\fcallg}[1]{\hat{g}_{\mathrm{call}} \parent{#1}}
\newcommand{\fcallga}[1]{\hat{g}_{\mathrm{call}, \damp} \parent{#1}}
\newcommand{\fputga}[1]{\hat{g}_{\mathrm{put}, \damp} \parent{#1}}
\newcommand{\fcallf}[1]{\hat{f}_{\mathrm{call}} \parent{#1}}
\newcommand{\fcallfa}[1]{\hat{f}_{\mathrm{call}, \damp} \parent{#1}}


\newcommand{\evaldiff}[3]{{\left [ {#1} \right ]}_{#2}^{#3}} 

\newcommand{\vga}{a}
\newcommand{\vgb}{b}

\newcommand{\vgpoly}[1]{w  \parent{#1}}
\newcommand{\vgaux}{q}

\newcommand{\imo}{y}
\newcommand{\reo}{x}
\newcommand{\abex}{\rho}

%%% for the merton case
\newcommand{\sj}{\sigma_{\mathrm{jump}}}
\newcommand{\rj}{r_{\mathrm{jump}}}


%% I always forget the correct spelling
\newcommand{\levy}{L\'evy }
\newcommand{\levynospace}{L\'evy}
%%

\newcommand{\rz}{\R \setminus \sset{0}}

\newcommand{\linf}[1]{L^\infty_{#1}}
\newcommand{\norminf}[2]{\left\|#1\right\|_{\linf{#2}}}
\newcommand{\cl}{C \parent{\lambda}}

\pgfmathdeclarefunction{lg10}{1}{%
    \pgfmathparse{ln(#1)/ln(10)}%
}


\title[MLMC for timer options]{Multi-Level Monte Carlo evaluation of timer options.}
\author{Juho H\"app\"ol\"a}
\address{\kaustaddress}
\email{juho.happola@iki.fi}


\date{\today}    
                                    % Activate to display a given date or no date
\begin{document}


\begin{abstract}
Abstract here!!!
\end{abstract}

\maketitle
%\tableofcontents


\section{Introduction}

Timer options are an interesting class of derivative assets that have gained some popularity recently. The central feature of timer options is their stochastic exercise time.

Let the asset price be modelled by the following SDE
\begin{align}
dX_t &= r_t X dt + \beta \parent{t,X_t, \sigma_t} dW_t
\nonumber \\
X_0 &= x_0
\end{align}
with the volatility $\sigma_t$, and interest rate $r_t$
given by adapted processes.

Unlike in plain vanilla options, we are interested in evaluating
the expected value of a payoff functional $g$
\begin{align}
\expp{e^{-\int_0^\tau r_s ds} g \parent{X_\tau}}
\end{align}
with the hitting time $\tau$ being defined as
\begin{align}
\label{eq:hittingTimeDefinition}
\tau \equiv \mathop{\mathrm{inf.}}_{\iota}
\sset{\iota: \int_0^\iota \sigma_s^2 ds \geq V}.
\end{align}

We note, that for processes that have almost surely $\sigma_s=\sigma$,
the option prices reduce to those of a fixed-maturity plain-vanilla option.

The case of stochastic volatility has been studied earlier
in the Forward-Euler Monte Carlo setting by \cite{bernard2011pricing}. For a selected class of stochastic
volatility models, \cite{zeng2015fast} presented a fast Hilbert transform based solution.

\section{Model}

For a rather general study framework of studying the pricing of timer options, we choose the Sch\"obel-Zhu-Hull-White model introduced in \cite{grzelak2012extension}, with stochastic interest rates and volatility:
\begin{align}
d S_t &=& r_t S_t dt &+& \sigma_t^p S_t dW_t^x 
\nonumber
\\
d r_t &=& \lambda \parent{\theta_t -r_t} dt  &+& \eta dW_t^r
\\
d \sigma_t &=& \kappa \parent{\overline \sigma - \sigma_t} dt
 &+& \gamma \sigma^{1-p} dW_t^\sigma
\nonumber,
\end{align}
where the three Brownian motions $W^x$, $W^r$ and $W^\sigma$
are correlated through the constant instantaneous correlation matrix
$\Sigma$.


\section{Numerical example}

To set a reasonable test case in which to test the convergence of
Multi-Level Monte Carlo, we may select the a reasonable asset volatility
to twenty per cent:
\begin{align*}
\overline \sigma = \sigma_0 = 0.2.
\end{align*}
Setting $p=1$, the
long-term variance of volatility is given by
\begin{align*}
\frac{\gamma^2 }{2 \kappa} = \frac{\bar \sigma}{4},
\end{align*}
we set
\begin{align*}
\kappa &= 2, \\
\bar \sigma &= \gamma.
\end{align*}
Likewise, for the interest rate, we set 
\begin{align*}
\theta_t &= 0.01, \\
\lambda &=1,
\eta = 0.1.
\end{align*}

We may select the maturity time $T=\frac{1}{12}$ and which
gives us
\begin{align*}
V = \frac{\bar \sigma^2}{12}.
\end{align*}

For the discretisation, we select the drift-implicit Euler method
which guarantees stability for longer time steps when using longer
time steps for the Ornstein-Uhlenbeck process.
\begin{align}
\sigma_{n+1} &= \frac{\kappa \overline \sigma}{1+\Delta t}
+ \gamma \sigma_{n}^{1-p} \Delta W_n^{\sigma}
\\
r_{n+1} &= \frac{\lambda \overline \theta_n}{1+ \Delta t}
+ \eta \Delta W_n^r
\\
\log X_{n+1} &=
 \parent{r - \frac{\sigma^{2p}_{n+1}}{2}} \Delta t
+ \sigma_n^{p} \Delta W_n^x .
\end{align}
To evaluate the accumulated variance, we discretise the integral in eq. \eqref{eq:hittingTimeDefinition} as
\begin{align}
\nu_{n+1} = \nu_n + \sigma_n^2 \Delta t,
\end{align}
and similarly for the discounting process
\begin{align}
d_{n+1} = d_n + r_n \Delta t.
\end{align}

For the correlation matrix, we initially set $\Sigma = \mathbf 1_3$.

Given a multi-level index $\ell \geq 0$, we set
\begin{align*}
\Delta t^{\ell} = 2^{-\ell-5} T,
\end{align*}
i.e. We start with a setting that for $\gamma=0$ case
has $2^5$ time steps and increase the number of time steps geometrically.

We simulate realisations $\overline X^\ell$, $\overline X^{\ell-1}$ for
$1<\ell<L$ in pairs using the same brownian paths and run the discretised Euler integration until $n^*,k^*$ $\nu_{n^*}^\ell \geq V$ and $\nu_{k^*}^{\ell+1} \geq V$. Finally, we form the estimator for the option price as
\begin{align}
\mathcal A \parent{g} = \ssum{m=0}{M_0} \frac{e^{d_{n^*}^{\parent{0}}} g \parent{X_{k^*}} }{M_0}
+
\ssum{\ell=1}{L} \ssum{m=0}{M_\ell}
\frac{e^{d_{n^*}^{\parent{0}}} g \parent{X_{n^*}} - e^{d_{k^*}^{\parent{0}}} g \parent{X_{k^*}} }{M_0}
\end{align}

\section{Convergence results (???)}

\bibliographystyle{agsm}
\bibliography{references}

\appendix




\end{document}


