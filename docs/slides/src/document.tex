{
\nofooter \noheader \frame[noframenumbering]{\titlepage}
}

\begin{frame}{Vision (the ambitious version)}
  \begin{itemize}
  \item Provide an easy to use, easy to extend open source
    library for UQ problem, both forward and inverse,
  \item Multilevel and Mult-index versions of Monte Carlo, Quasi Monte
    Carlo, Stochastic collocation, Least square projection among
    others.
  \item Support parallel computation whenever possible.
  \item Provide easy to use storage facility.
  \item Provide easy to customize plotting facility (for common plots).
  \item Use \texttt{Python} for easier implementation of most parts of code
    and use object code (\texttt{C++} or \texttt{FORTRAN}) for
    computationally expensive parts.
  \end{itemize}
\end{frame}

\begin{frame}{What has been done}
  \begin{itemize}
  \item Multilevel and Mult-index versions of Monte Carlo
  \item MySQL database storage.
  \item Still need to port existing plotting code
  \item Needs heavy documenting
  \item Needs further testing
  \end{itemize}
\end{frame}

\begin{frame}[fragile]{Installation (for users)}
\begin{verbatim}
> cd WorkDir
> git clone \
https://ahajiali@bitbucket.org/ahajiali/mimclib.git
> cd mimclib
> make

> python -c 'from  mimclib.db import DBCreationScript ; \
print DBCreationScript();' | mysql
\end{verbatim}
\end{frame}

\begin{frame}[fragile]{A typical python example for a single MLMC run}
\begin{verbatim}
# Read arguments from command line
import argparse
parser = argparse.ArgumentParser(add_help=True)
mimc.MIMCRun.addOptionsToParser(parser)
tmp = parser.parse_known_args()[0]
mimcRun = mimc.MIMCRun(**vars(tmp))

# Create entry in DB for MIMCRun
db = mimcdb.MIMCDatabase()
run_id = db.createRun(TOL=mimcRun.params.TOL,
                      tag="MyFirstMIMCRun",
                      dim=mimcRun.data.dim,
                      params=mimcRun.params)
\end{verbatim}
\end{frame}

\begin{frame}[fragile]{User defined functions with typical impl.}
\begin{verbatim}
def fnWorkModel(lvls):
  return mimc.work_estimate(lvls, mimcRun.params.gamma)

def fnHierarchy(lvls):
  return mimc.get_geometric_hl(lvls,
                               mimcRun.params.h0inv,
                               mimcRun.params.beta)

def fnItrDone(itrIndex, TOL, totalTime)
  db.writeRunData(run_id, mimcRun,
                  itrIndex, TOL, totalTime)

def fnSampleLvl(moments, mods, inds, M):
  ...


\end{verbatim}
\end{frame}


\begin{frame}[fragile]{A typical python example for a single MLMC run}
\begin{verbatim}
mimcRun.setFunctions(fnWorkModel=fnWorkModel,
                     fnHierarchy=fnHierarchy,
                     fnSampleLvl=fnSampleLvl,
                     fnItrDone=fnItrDone)

mimcRun.doRun()
print("Final value:", mimcRun.data.calcEg())
\end{verbatim}
\end{frame}

\begin{frame}[fragile]{Running the script. MLMC}
\begin{verbatim}
> python run.py --  -mimc_TOL 0.001 \
   -mimc_verbose True -mimc_dim 1 -mimc_bayesian True \
   -mimc_w 2 -mimc_s 4 -mimc_gamma 3 \
   -mimc_beta 2
\end{verbatim}
\end{frame}

\begin{frame}[fragile]{Running the script. MLMC}
\begin{verbatim}
> python run.py --  -mimc_TOL 0.001 \
   -mimc_verbose True -mimc_dim 3 -mimc_w 2 2 2 \
   -mimc_s 4 4 4 -mimc_gamma 1 1 1 \
   -mimc_beta 2 2 2
\end{verbatim}
\end{frame}

%%% Local Variables:
%%% mode: latex
%%% TeX-master: "../main"
%%% End:
